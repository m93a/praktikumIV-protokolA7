% !TEX program = xelatex

% Nejlepší zážitek zaručí:
%
% TeX distribuce: texlive-full
%
% Editor:
%   VS Code s doplňky
%       * LaTeX Workshop
%       * LaTeX Utilities
%       * Gnuplot
%
% Další závislosti:
%   latexmk
%   bibtex
%   gnuplot


% Jak používat:
% Zkompilovat: make
% Gnuplot: make gnuplot
% Vyčistit: make clean


% Základní balíčky
\documentclass[10pt,a4paper]{article}
\usepackage[utf8]{inputenc}
\usepackage[czech]{babel}
\usepackage{graphicx}
\usepackage{lmodern}
\usepackage{amsmath}
\usepackage{hyperref}
\usepackage{gensymb}
\usepackage[top = 2cm, bottom = 2cm, left = 2cm, right = 2cm]{geometry}

% Pro titulní stránku
\usepackage{titlesec}
\usepackage{setspace}
\usepackage{framed}
\usepackage{array}

% Vlastní balíčky 
\usepackage{gnuplottex}
\usepackage{epstopdf}
\usepackage{csvsimple}


\renewcommand{\U}[1]{\ensuremath{\,\mathrm{#1}}}
\newcommand{\°}{\degree}

\newcommand{\titjmeno}{Michal Grňo}
\newcommand{\titobor}{FOF}


\newcommand{\titcislo}{A7}
\newcommand{\titnazev}{Pozitronová emisní tomografie}
\newcommand{\titmereni}{7. 10. 2019}
\newcommand{\titodevzdani}{20. 10. 2019}


\begin{document}

\include{titulka}

\section{Pracovní úkoly}
\begin{enumerate}
    \item Poté, co vyučující umístí silnější zářič ${}^{22}$Na do stojánku, změřte úhlové rozdělení koincidencí v oblasti úhlů potřebné pro nalezení polohy zářiče, doba měření 20s. Vysvětlete tvar naměřeného úhlového rozdělení, získané poznatky využijte při domácím zpracování.

    \item Změřte četnost koincidencí pro úhly $\varphi$ = 60°, 90°, 120° bez plechu a 120° s Pb plechem mezi detektory, doba měření 100s. Vysvětlete pozorované četnosti.
    
    \item Poté, co vyučující přidá do krabičky druhý zářič, změřte úhlové rozdělení koincidencí s krokem 5°.
    
    \item Zvolte aspoň 2 další vhodné úhly otočení krabičky $\psi$ a opakujte měření 3).
    
    \item Narýsujte přímky spojující detektory do obrázku připraveného u úlohy a odečtěte polohu průsečíku - polohu zářiče vůči krabičce. Pozn.: Při volbě otočení krabičky $\psi$ se můžete řídit polohou už zakreslených průsečíků.
    
    \item Vzdálenost detektoru od zářiče zakresleného na obrázku porovnejte s měřením skutečné vzdálenosti.
    
    \item Polohy zářičů vůči krabičce určujte pomocí vztahů a metod popsaných v návodu. Podle výsledků zpracování nakreslete obrázky analogické k obrázkům narýsoaným během praktika. Chyby polohy zářičů určete graficky
\end{enumerate}


\section{Teoretická část}
Účelem práce je použít pozitronovou emisní tomografii (PET) k určení umístění radioaktivního vzorku uvnitř modelu lebky.
PET využívá radionuklidů s $\beta^+$ rozpadem, tedy takových, které při rozpadu produkují pozitron – ten se po rozpadu v okolní látce zpomalí téměř do klidu (na vzdálenosti řádu $1 \U{mm}$) a potom anihiluje s nějakým elektronem. Výsledkem anihilace jsou dva fotony letící téměř přesně opačným směrem. \cite{studijni-text}

Po obvodu kolem krabičky se vzorky jsou umístěny dva detektory v koincidenčním zapojení – tzn. že zaznamenají pouze, když do obou přiletí foton téměř zároveň. Je žádoucí, aby byl časový interval mezi příchozími fotony co nejmenší. Současná technika je schopna dosáhnout intervalů $10^{-6}$ až $10^{-12} \U{s}$. \cite{studijni-text} Čím delší je interval, tím je aparatura náchylnější na šum.

\begin{figure}[h]
    \centering
    \includegraphics[width=10cm]{schema.png}
    \caption{Schéma koincidenčního měření, převzato z [1].}
\end{figure}

V našem experimentu byl jeden detektor (A) nepohyblivý a druhým detektorem (B) bylo možné pohybovat po kružnici se středem uprostřed vzorku. Polohu detektoru B popisujeme úhlem $\varphi$ měřeným v kladném směru od polohy naproti detektoru A. Oba detektory jsou umístěny ve vzdálenosti $R$ od středu vzorku. Samotným vzorkem lze otáčet kolem středu, jeho otočení popisujeme úhlem $\psi$.

Vztažnou sosutavu vzorku popisujeme souřadnicemi $x, y$. V této souřadné soustavě budou mít detektory A a B souřadnice:
\begin{gather}
    \begin{aligned}
        x_A &= -R \cos \psi & x_B &= R \cos(\psi + \varphi)\\
        y_A &= -R \sin \psi & y_B &= R \sin(\psi + \varphi)
    \end{aligned}
    \label{detektory-xy}
\end{gather}
Přímka mezi detektory, na které bude ležet zářič, je popsaná rovnicí
\begin{equation}
    (y_b - y_a)x + (x_a - x_b)y = (y_b - y_a)x_a + (x_a - x_b)y_a
    \label{primka-xy}
\end{equation}


\section{Výsledky měření}
Nejprve jsme s $t=20\U{s}$ měřili koincidence pro vzorek s jedním zářičem. Pro pevné hodnoty $\psi$ jsme proměřili několik hodnot $\varphi$, naměřené hodnoty jsou v grafu na obr. č. \ref{graf-1vz}. Naměřenými daty byla pomocí metody nejmenších čtverců\footnote{Teoreticky správné by bylo použít Poissonovu regresi, metoda nejmenších čtverců je ale stále dobrou aproximací.} proložena gaussova křivka. Maxima pro jednotlivé hodnoty $\psi$ i s chybou fitu jsou v tabulce č. \ref{tab-fity-1vz}.


\begin{table}[h!]
    \centering
    \begin{tabular}{r|r|r}
        % Hlavička
        \bfseries $\psi$ &
        \bfseries $\varphi$ &
        \bfseries $\Delta\varphi$

        % Soubor
        \csvreader[ head to column names ]{fits1.csv.tmp}{}
        {
            \csviffirstrow{\\\hline}{\\} \psival\ & \phival & \phierr
        }
    \end{tabular}

    \caption{Úhly získané regresí, 1 zářič}
    \label{tab-fity-1vz}
\end{table}



\begin{figure}[p]
    \centering
    \begin{gnuplot}[terminal=epslatex,terminaloptions=color]

        set xlabel "$\\varphi$ [\°]"
        set ylabel "N"
        load 'lines.cfg'
        set macros
        set key left top

        nd(A, x0, sigma, x) = A * exp(-(x-x0)**2/2./sigma**2)


        colselector(phi) = \
            '( $1 > 180 ? $1 - 360 : $1 ):($3 == '.phi.' ? $2 : 1/0)'
        
        set1 = colselector(0)
        set2 = colselector(30)
        set3 = colselector(60)
        set4 = colselector(90)

        A1 = A2 = A3 = A4 = 400
        mean1 = mean2 = mean3 = mean4 = 1
        sigma1 = sigma2 = sigma3 = sigma4 = 5


        f1(x) = nd(A1, mean1, sigma1, x)
        f2(x) = nd(A2, mean2, sigma2, x)
        f3(x) = nd(A3, mean3, sigma3, x)
        f4(x) = nd(A4, mean4, sigma4, x)


        fit f1(x) 'data1' using @set1 via A1, mean1, sigma1
        fit f2(x) 'data1' using @set2 via A2, mean2, sigma2
        fit f3(x) 'data1' using @set3 via A3, mean3, sigma3
        fit f4(x) 'data1' using @set4 via A4, mean4, sigma4


        set print "fits1.csv.tmp"
        print "psival, phival, phierr"
        print sprintf( "0, %.2f, %.2f", mean1, mean1_err)
        print sprintf("30, %.2f, %.2f", mean2, mean2_err)
        print sprintf("60, %.2f, %.2f", mean3, mean3_err)
        print sprintf("90, %.2f, %.2f", mean4, mean4_err)


        plot \
            'data1' using @set1 notitle lc palette cb 0, \
            f1(x) notitle lc palette cb 0, \
            NaN w lp title "$\\psi$ = 0°" lc palette cb 0, \
            \
            'data1' using @set2 notitle lc palette cb 2, \
            f2(x) notitle lc palette cb 2, \
            NaN w lp title "$\\psi$ = 30°" lc palette cb 2, \
            \
            'data1' using @set3 notitle lc palette cb 3, \
            f3(x) notitle lc palette cb 3, \
            NaN w lp title "$\\psi$ = 60°" lc palette cb 3, \
            \
            'data1' using @set4 notitle lc palette cb 4, \
            f4(x) notitle lc palette cb 4, \
            NaN w lp title "$\\psi$ = 90°" lc palette cb 4

    \end{gnuplot}

    \caption{Naměřené koincidence, 1 zářič}
    \label{graf-1vz}
\end{figure}


Následně jsme pro fixní $\psi=0\°$ proměřili koincidence při $t=100\U{s}$ na $\varphi=60\°, 90\°, 120\°$. Poté jsme oba detektory zastínili olověnou deskou a znovu změřili počet koincidencí při $\varphi=120\°$. Naměřené výsledky jsou v grafu na obr. č. \ref{nestin-vs-stin}. Z dat je zřejmé, že šum způsobený samotnou aparaturou (tj. hodnota naměřená po odstínění) je výrazně nižší, než šum přicházející od vzorku. Mírný růst počtu koincidencí při přibližování detektorů může být buď důsledkem Comptonova jevu, anebo čistě náhodný.

\begin{figure}[p]
    \centering
    \begin{gnuplot}[terminal=epslatex,terminaloptions=color]
        
        set xlabel "$\\varphi$ [\°]"
        set ylabel "N"
        set xrange [40:140]
        set yrange [0:80]
        load 'lines.cfg'
        set key left bottom
        set key box vertical width 1 height 1 maxcols 0.3 spacing 1

        f(x) = a*x + b
        fit f(x) 'data2' using 1:($3 == 0 ? $2 : 1/0) via a, b

        plot \
            'data2' using 1:($3 == 0 ? $2 : 1/0) \
                ps 3 lc palette cb 0 title "nestíněný", \
            \
            'data2' using 1:($3 == 1 ? $2 : 1/0) \
                ps 3 lc palette cb 1 title "stíněný", \
            \
            f(x) lc palette cb 0 dashtype ' _ ' notitle

    \end{gnuplot}

    \caption{Nestíněné vs. stíněné detektory}
    \label{nestin-vs-stin}
\end{figure}

Poté jsme odebrali olověnou desku a do vzorku přidali druhý zářič. Pro každou hodnotu $\psi = 0\°, 60\°, 90\°$ jsme při $t=20\U{s}$ opět proměřili počet koincidencí pro různá $\varphi$. Naměřené hodnoty jsou v grafu na obr. č. \ref{graf-2vz}. Data jsme proložili součtem dvou gaussových křivek, tím získané peaky $\varphi_1$ a $\varphi_2$ jsou v tabulce č. \ref{tab-fity-2vz}.

\begin{table}[h!]
    \centering
    \begin{tabular}{r|r|r|r|r}
        % Hlavička
        \bfseries $\psi$ &
        \bfseries $\varphi_1$ &
        \bfseries $\Delta\varphi_1$ &
        \bfseries $\varphi_2$ &
        \bfseries $\Delta\varphi_2$

        % Soubor
        \csvreader[ head to column names ]{fits3.csv.tmp}{}
        {
            \csviffirstrow{\\\hline}{\\}
            \psival\ &
            \phiA & \phiAerr &
            \phiB & \phiBerr
        }
    \end{tabular}
    
    \caption{Úhly získané regresí, 2 zářiče}
    \label{tab-fity-2vz}
\end{table}

\begin{figure}[p]
    \centering
    \begin{gnuplot}[terminal=epslatex,terminaloptions=color]

        set xlabel "$\\varphi$ [\°]"
        set ylabel "N"
        load 'lines.cfg'
        set macros
        set key left top

        arbitrarilyLargeNumber = 1000000000

        nd(A, x0, sigma, x) = A * exp(-(x-x0)**2/2./sigma**2)


        colselector(phi) = \
            '( $1 > 180 ? $1 - 360 : $1 ):($3 == '.phi.' ? $2 : 1/0)'
        
        set1 = colselector(0)
        set2 = colselector(60)
        set3 = colselector(90)

        A1a = A1b = A2a = A2b = A3a = A3b = 400
        mean1a = mean2a = mean3a = mean4a = -20
        mean1b = mean2b = mean3b = mean4b = 20
        sigma1a = sigma2a = sigma3a = sigma4a = 15
        sigma1b = sigma2b = sigma3b = sigma4b = 15
        noise1 = noise2 = noise3 = 1

        f1(x) = noise1 < 0 ? arbitrarilyLargeNumber : \
            nd(A1a, mean1a, sigma1a, x) + nd(A1b, mean1b, sigma1b, x) + noise1

        f2(x) = noise2 < 0 ? arbitrarilyLargeNumber : \
            nd(A2a, mean2a, sigma2a, x) + nd(A2b, mean2b, sigma2b, x) + noise2

        f3(x) = noise3 < 0 ? arbitrarilyLargeNumber : \
            nd(A3a, mean3a, sigma3a, x) + nd(A3b, mean3b, sigma3b, x) + noise3


        fit f1(x) 'data3' using @set1 \
        via A1a, mean1a, sigma1a, A1b, mean1b, sigma1b, noise1

        fit f2(x) 'data3' using @set2 \
        via A2a, mean2a, sigma2a, A2b, mean2b, sigma2b, noise2

        fit f3(x) 'data3' using @set3 \
        via A3a, mean3a, sigma3a, A3b, mean3b, sigma3b, noise3


        set print "fits3.csv.tmp"
        print "psival, phiA, phiAerr, phiB, phiBerr"
        fmt = "%d, %.2f, %.2f, %.2f, %.2f"
        print sprintf(fmt,  0, mean1a, mean1a_err, mean1b, mean1b_err)
        print sprintf(fmt, 60, mean2a, mean2a_err, mean2b, mean2b_err)
        print sprintf(fmt, 90, mean3a, mean3a_err, mean3b, mean3b_err)


        plot \
            'data3' using @set1 notitle lc palette cb 0, \
            f1(x) notitle lc palette cb 0, \
            NaN w lp title "$\\psi$ = 0°" lc palette cb 0, \
            \
            'data3' using @set2 notitle lc palette cb 2, \
            f2(x) notitle lc palette cb 2, \
            NaN w lp title "$\\psi$ = 60°" lc palette cb 2, \
            \
            'data3' using @set3 notitle lc palette cb 3, \
            f3(x) notitle lc palette cb 3, \
            NaN w lp title "$\\psi$ = 90°" lc palette cb 3

    \end{gnuplot}

    \caption{Naměřené koincidence, 2 zářiče}
    \label{graf-2vz}
\end{figure}

\begin{figure}[p]
    \centering
    \begin{gnuplot}[terminal=epslatex,terminaloptions=color]

        load 'matlib.cfg'
        load 'vzorce.cfg'



        # načti první soubor

        LF_File = 'fits1.csv.tmp'
        LF_Columns = 3
        load 'loadfile.cfg'

        m = sliceArray('LF_Col1','psi',1,0)
        @m
        m = sliceArray('LF_Col2','phi',1,0)
        @m

        do for [i=1:|psi|] { psi[i] = real(psi[i]) }
        do for [i=1:|phi|] { phi[i] = real(phi[i]) }


        # vypočítej průsečíky prvního

        N = nCr(|psi|, 2)
        array xintA[N]
        array yintA[N]

        i = 1
        do for [j=1:|psi|] {
            do for [k=1:j-1] {
                xintA[i] = xintf(psi[j], phi[j], psi[k], phi[k])
                yintA[i] = yintf(psi[j], phi[j], psi[k], phi[k])
                i = i + 1
            }
        }

        plnA = ""
        defA = ""
        do for [i=1:|psi|] {
            defA = defA."; ".line("lineA".i, psi[i], phi[i])
            plnA = plnA.", lineA".i."(x) notitle"
        }
        @defA


        # načti druhý soubor

        LF_File = 'fits3.csv.tmp'
        LF_Columns = 5
        load 'loadfile.cfg'

        n = |LF_Col1| - 1
        N = 2 * n

        array psi[N]
        array phi[N]

        do for [i=1:n] {
            psi[i]   = real(LF_Col1[i+1])
            psi[i+n] = real(LF_Col1[i+1])
        }

        do for [i=1:n] { phi[i]   = real(LF_Col2[i+1]) }
        do for [i=1:n] { phi[i+n] = real(LF_Col4[i+1]) }



        # vypočítej průsečíky druhého

        N = nCr(|psi|, 2)
        array xintB[N]
        array yintB[N]

        i = 1
        do for [j=1:|psi|] {
            do for [k=1:j-1] {
                xintB[i] = xintf(psi[j], phi[j], psi[k], phi[k])
                yintB[i] = yintf(psi[j], phi[j], psi[k], phi[k])
                i = i + 1
            }
        }

        plnB = ""
        defB = ""
        do for [i=1:|psi|] {
            defB = defB."; ".line("lineB".i, psi[i], phi[i])
            plnB = plnB.", lineB".i."(x) notitle"
        }
        @defB


        #vykresli

        set multiplot layout 1, 2

        set xlabel "x"
        set ylabel "y"

        set size square
        set xrange [-0.5:0.5]
        set yrange [-0.5:0.5]
        
        set title "1 zářič"
        plot \
            sample [i=1:|xintA|] '+' using (xintA[i]):(yintA[i]) notitle \
            @plnA

        unset ylabel
        
        set title "2 zářiče"
        plot \
            sample [i=1:|xintB|] '+' using (xintB[i]):(yintB[i]) notitle \
            @plnB

        unset multiplot


        # zapiš do souboru

        set terminal dumb

        f(x) = -(0.7*x + 0.1)/0.2
        g(x) = f(x) + 0.7
        h(x) = x

        conditionA(x, y) = y > f(x) && y < g(x) && y < h(x)
        conditionB(x, y) = y > f(x) && y < g(x) && y > h(x)
        
        set print 'int1.csv.tmp'
        print "x, y"

        do for [i=1:|xintA|] {
            print sprintf("%f, %f", xintA[i], yintA[i])
        }

        set print 'int2A.csv.tmp'
        print "x, y"

        do for [i=1:|xintB|] {
            if (conditionA(xintB[i], yintB[i])) {
                print sprintf("%f, %f", xintB[i], yintB[i])
            }
        }

        set print 'int2B.csv.tmp'
        print "x, y"

        do for [i=1:|xintB|] {
            if (conditionB(xintB[i], yintB[i])) {
                print sprintf("%f, %f", xintB[i], yintB[i])
            }
        }
        
    \end{gnuplot}

    \caption{Průsečíky vypočtených přímek}
    \label{graf-pruseciky}
\end{figure}

Podle údajů v tabulkách \ref{tab-fity-1vz} a \ref{tab-fity-2vz} jsme určili rovnice přímek, na kterých musí ležet zářiče, tyto přímky jsme vynesli do grafů na obr. č. \ref{graf-pruseciky}. Na grafu 2 zářičů lze pozorovat i falešné průsečíky, které vznikají mezi přímkami pocházejícími od různých zářičů. Ze souřadnic průsečíků jsme vypočetli průměrné souřadnice a chybu průměru. Ty jsou v tabulce č. \ref{tab-average-pts}.

\begin{gnuplot}[terminal=epslatex,terminaloptions=color]

    stats 'int1.csv.tmp'
    x1 = STATS_mean_x
    y1 = STATS_mean_y
    x1err = STATS_mean_err_x
    y1err = STATS_mean_err_y

    stats 'int2A.csv.tmp'
    x2A = STATS_mean_x
    y2A = STATS_mean_y
    x2Aerr = STATS_mean_err_x
    y2Aerr = STATS_mean_err_y

    stats 'int2B.csv.tmp'
    x2B = STATS_mean_x
    y2B = STATS_mean_y
    x2Berr = STATS_mean_err_x
    y2Berr = STATS_mean_err_y

    set print "average_pts.csv.tmp"
    print "x, xerr, y, yerr, note"
    print sprintf("%.3f, %.3f, %.3f, %.3f, Jednozáříčový setup", x1, x1err, y1, y1err)
    print sprintf("%.3f, %.3f, %.3f, %.3f, Dvouzářičový setup; zářič A", x2A, x2Aerr, y2A, y2Aerr)
    print sprintf("%.3f, %.3f, %.3f, %.3f, Dvouzářičový setup; zářič B", x2B, x2Berr, y2B, y2Berr)

\end{gnuplot}

\begin{table}[h!]
    \centering
    \begin{tabular}{r|r|r|r|l}
        % Hlavička
        \bfseries $x$ &
        \bfseries $\Delta x$ &
        \bfseries $y$ &
        \bfseries $\Delta y$ &
        \bfseries pozn.

        % Soubor
        \csvreader[ head to column names ]{average_pts.csv.tmp}{}
        {
            \csviffirstrow{\\\hline}{\\}
            \x & \xerr &
            \y & \yerr &
            \note
        }
    \end{tabular}
    
    \caption{Průměrné polohy středů zářičů}
    \label{tab-average-pts}
\end{table}

\pagebreak

\section{Diskuse}
Při zpracování dat bylo využito předpokladu, že při $\psi = \mathrm{konst.}$ odpovídá vztah $N(\varphi)$ gaussově křivce, a že lze chybu měření $N$ aproximovat normální distribucí. Z grafů vidíme, že křivky dobře kopírují naměřená data, předpoklady tedy zjevně byly splněny.

Ačkoliv by měl zářič v jednozářičovém setupu odpovídat zářiči B ve dvouzářičovém, vidíme, že se signifikantně liší. To bylo pravděpodobně způsobeno pootočením vzorku mezi měřeními (stejný zaznamenaný úhel $\psi$ tedy odpovídal jinému natočení).


\section{Závěr}
Podařilo se proměřit koincidence pro různé úhly, z nich se podařilo přesně rekonstruovat polohu zářičů uvnitř vzorku (tabulka \ref{tab-average-pts}). Nepodařilo se ověřit přesnost měření tak, že by jeden zářič zůstal konstantní při obou měřeních.

\section{Literatura}
\bibliography{literatura} 
\bibliographystyle{ieeetr}

 
\end{document}